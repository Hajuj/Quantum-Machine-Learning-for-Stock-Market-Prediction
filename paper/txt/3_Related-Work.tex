A considerable amount of research has investigated financial forecasting using computational approaches, particularly focusing on classical machine learning algorithms trained for time series prediction \cite{DEGOOIJER2006443}. Notably, RNNs and their variants, such as LSTM or Gated Recurrent Unit, have demonstrated to be especially promising in this context \cite{SEZER2020106181} \cite{SAGHEER2019203} \cite{10.1145/3377713.3377722}.

Although numerous studies have highlighted the exceptional performance of these classical algorithms, this paper shifts its focus towards exploring Quantum Neural Networks (QNNs) as a potential way for surpassing the capabilities of classical neural networks. Quantum machine learning is currently a extensively researched area, with classical models being successfully translated into their quantum counterparts. Notably, the domain of QRNNs has witnessed significant attention, going beyond the mere translation of simple RNNs \cite{chen2020quantum} \cite{bausch2020recurrent}.

In related studies, researchers have addressed challenges akin to financial forecasting. For instance, in the work by Li et al. \cite{li2023quantum}, a novel implementation of the QRNN is applied for sequential learning and assessed in the context of meteorological forecasting, text categorization, and stock price prediction. Comparative analyses against the classical counterpart of the QRNN reveal a substantial improvement in prediction accuracy. Furthermore, when contrasted with an alternative QNN, the performance of the QRNN surpasses that of the QNN, underscoring its enhanced predictive capabilities.

Moreover, existing research has explored solutions to the same problem under consideration in this paper by employing quantum neural networks. For instance, the work by Emmanoulopoulos et al. \cite{emmanoulopoulos2022quantum} addresses this challenge by comparing the efficacy of an implemented quantum neural network against that of a classical bidirectional long short-term memory neural network (BiLSTM). The outcomes, consistent with those discussed previously, demonstrate that their parameterized quantum circuit performs comparably to a classical BiLSTM, despite requiring significantly fewer parameters.

Similarly, noteworthy is the methodology presented in this work \cite{s2023potential}, wherein a Quantum Support Vector Machine is employed for forecasting future stock closing prices, deviating from the conventional use of QNNs. A distinctive feature of this approach is the incorporation of diverse stock price indicators, including Moving Averages, Average True Range, and Aroon, as inputs, offering a more comprehensive understanding of the stock market dynamics beyond relying solely on past stock closing prices.

Moreover, significant strides have been made in the fundamental research aimed at transposing LSTM architectures into the quantum domain, as elucidated by the work of the following paper \cite{chen2020quantum}. In this seminal contribution, the authors explore a novel hybrid quantum-classical LSTM model. After its development, the model undergoes  testing on mathematical functions, revealing substantial performance enhancements.

Building upon this foundational research, our work seeks to extend the applicability of such QLSTM models. Drawing inspiration from the aforementioned study, we aim to employ a QLSTM model for the analysis of time series data. By doing so, we intend to ascertain the efficacy of this model in the realm of time series forecasting, thereby contributing to the ongoing exploration of quantum-assisted methodologies in the field. The objective is to harness the demonstrated benefits of the hybrid model in mathematical functions and investigate its potential advantages when applied to the intricate domain of time series analysis.